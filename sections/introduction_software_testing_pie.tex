%!TEX root=kapfhammer_gcc_presentation.tex
% mainfile: ../kapfhammer_gcc_presentation.tex

\begin{frame}[t]

  \frametitle{The PIE Model}
  \framesubtitle{There are necessary and sufficient conditions for fault detection}

  \begin{tikzpicture}[overlay, remember picture]
    \node[anchor=center] at (current page.center) {
        \begin{beamercolorbox}[center]{title}
          \vspace*{.75in}
          \Large
          \begin{center}

            \only<1-3>{
              \begin{itemize}
                \item<1-> Execute the faulty source code
                \item<2-> Infect the program's data state
                \item<3-> Propagate to the program's output
              \end{itemize}
            }

            \only<4->{
              \begin{itemize}
                \item {\color{solarizedYellow}E}xecute the faulty source code
                \item {\color{solarizedYellow}I}nfect the program's data state
                \item {\color{solarizedYellow}P}ropagate to the program's output
              \end{itemize}
            }

            \only<1-4>{
              \vspace*{.25in}
              {\color{kapfhammerDarkGrey}{\em All} of these {\em must occur} before the fault manifests itself as a failure!}
            }

            \only<5->{
              \vspace*{.25in}
              {\color{solarizedOrange}{\em All} of these {\em must occur} before the fault manifests itself as a failure!}
            }

            \only<1-5>{
              \vspace*{.25in}
              {\color{kapfhammerDarkGrey}Using PIE, will the test cases {\em find} the defect in the program?}
            }

            \only<6->{
              \vspace*{.25in}
              {\color{solarizedOrange}Using the PIE model, will the test cases {\em find} the defect in the program?}
            }

          \end{center}
          \normalsize
      \end{beamercolorbox}};
  \end{tikzpicture}

\end{frame}
